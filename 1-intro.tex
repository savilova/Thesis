
	5.8.1 Введение должно содержать оценку современного состояния решаемой научно-технической проблемы, основание и исходные данные для разработки темы, обоснование необходимости проведения НИР, сведения о планируемом научно-техническом уровне разработки, о патентных исследованиях и выводы из них, сведения о метрологическом обеспечении НИР. Во введении должны быть показаны актуальность и новизна темы, связь данной работы с другими научно-исследовательскими работами.

От быстрого протитипирования к аддитвному производству.

Главные преимущества СЛС перед другими технологиями, это 	[vaganov corrected]
:
\begin{itemize}
    \item свойства изделий, полученных методом СЛС схожи со свойствами изделий полученных традиционными методами, такими как extrusion, injection molding, hot pressing, etc.
    \item теоретически, в технологи СЛС может быть использова любой материал, который можно получить в порошке и расплавить при increasing temperature
    \item при печати сложых продуктов не нужно дополнительных опорных элементов.
\end{itemize}

	Researches on the application of other polymers in SLS technology are very limited. In addition to polyamide, polystyrene and polycarbonate are used as a starting material for SLS.12,13
	[vaganov corrected]
	
	"Анализ современной литературы показывает, что на сегодняшний день исследования в области применения термостойких полимеров для СЛС сосредоточе-ны лишь на семействе полиэфиркетонов [Berretta S., Ghita O., Evans K. E. // European Polymer Journal, 2014. Vol. 59. P. 218–229]. На данный момент практически отсутствуют сведения об использовании других клас-сов термостойких полимеров, таких, например, как полиимиды (ПИ). Кроме того, существует ряд проти-воречивых сведений о влиянии тех или иных параметров на процесс СЛС и конечные свойства материала в целом. В связи с этим, с научной и технической точки зрения весьма перспективным представляется иссле-дование закономерностей формирования структуры при СЛС порошков ПИ, изучение процесса их сплав-ления в зависимости от свойств порошка, содержания нанонаполнителей, а также параметров СЛС.
	В ИВС РАН синтезированы термопластичные частично кристаллические ПИ гомологического ряда Р-ОДФО на основе отечественного резорцинового диангидрида Р (1,3-бис-(3,3,4,4-дикарбоксифенокси)-бензол) и четырехядерного диамина ОДФО (4,4-бис(4-аминофенокси)бифенил) методом химической (с использованием катализаторов) и термической имидизации [Юдин В. Е., Светличный В. М. // Высоко-молекулярные соединения, серия С, 2016. Т. 58. №1. С. 19]. Исследованы форма, размер частиц (фракци-онный состав ПИ порошка). Показано, что методом химической имидизации формируются порошки с более узким распределением частиц по размерам (6–10 мкм), по сравнению с термически имидизованным порошком, а при увеличении молекулярной массы и дополнительной термообработки возрастает их насыпная плотность (до 0,5 г/мл), что способствует улучшению механических характеристик изделий, формируемых по методу СЛС. Важно отметить, что с увеличением молекулярной массы ПИ наблюдается резкое возрастание вязкости его расплава, что препят-ствует слиянию частиц в процессе СЛС.
На основе ПИ порошка Р-ОДФО впервые мето-дом СЛС получены образцы в виде пленок. Исследо-ваны свойства образцов в зависимости от способа синтеза, молекулярной массы и мощности лазера и показано, что более однородная и плотная структу-ра плёнок с относительно высокими механическими характеристиками получается из порошков, синтези-руемых по методу химической имидизации.
	"[ПОЛИМЕРНЫЕ МАТЕРИАЛЫ И ТЕХНОЛОГИИ Т.4 (2018), №3, 5
Редакционная колонка — личное мнение
Полиимидные порошки для 3D-печати по методу СЛС]
	
	
	***
	Таким образом, многообещающим направлением в СЛС является использование high-performance термостабильных термопластичых полимеров и модификация полимеров with small additives наночастиц.
[SYNTHESIS AND INVESTIGATION OF NANOMODIFED POLYIMIDE POWDER FOR
SELECTIVE LASER SINTERING
Vaganov G.V.1, Didenko A.L.1, Polyakov I.V.2, Ivan’kova E.M.1, Popova E.N.1, Yudin V.E.1,2 MODERN PROBLEMS
OF POLYMER SCIENCE
Program and Abstract Book
of 14th International Saint Petersburg Conference
of Young Scientists]	


нерешенные проблемы

актуальность

+цели и задачи
