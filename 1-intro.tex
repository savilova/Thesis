
%АКТУАЛЬНОСТЬ
% Про additive manufacturing

Аддитивное производство, также исвестное как 3D-печать, переводит виртуальные трехмерные модели в физические объекты. С помощью срезов CAD, трехмерного скана или данных томографии, АП строит объекты слой за слоем бещ помощи прессформ или обработки на станке.АП делает возможным децентрализованное производство объектов по индивидуальным требоаваниям, используя цифровое хранилище данных или интернет. 
Продолжающийся переход
от быстрого прототипирования к быстрому изготовлению ставит новые сложные задачи как перед инженерами-механиками, так и перед материаловедами.
Полимерны являются наиболее часто используемыми материалами для 3D-печати.
Ассортимент полимеров, используемых в АП, охватывает термопласты, реактопласты, эластомеры, гидрогели, функциональные полимеры,polymer blends, композиты и биологические системы.
\cite{3d-review}


С развитием аддитивных тезнологий от инструментов прототипирования к производству реальных деталей для конечного использования, возникает растущая потребность в возможности  обработки много большего числа различных материалов, чем возможно в настоящий момент. Материалы, наиболее часто используемые в 3D-печати, такие как нейлон-12, не удовлетворяет требованиям, предъявляемым к большинство коммерческих продуктов. Таким образом, как академические, так и индустриальные учреждения проявляют значительный интерес к увеличению ассортимента доступных материалов и исчерпывающему пониманию их фундаментальных свойств \cite{sls-material}.

Например, нужны термоустойчивые материалы, чтобы строить самолеты!

% Про отечественные аналоги
Таким материалом является, например, полиэфиримид Ultem-1000  Ввиду сами-знаете-чего, актуальна разработка и изучение отечественных аналогов. В ИВС РАН были синтезированы термопластичные частично кристаллические ПИ гомологического ряда Р-ОДФО на основе отечественного резорцинового диангидрида Р (1,3-бис-(3,3,4,4-дикарбоксифенокси)-бензол) и четырехядерного диамина ОДФО (4,4-бис(4-аминофенокси)бифенил) методом химической (с использованием катализаторов) и термической имидизации \cite{yudin-red}
% Про СЛС
СЛС - кратко [ССЫЛКА]

Применяется для индустрии (3д review)


Таким образом, многообещающим направлением в СЛС является использование  термостабильных термопластичых полимеров c высокими характеристиками и модификация полимеров with small additives наночастиц.
Главные преимущества СЛС перед другими технологиями, это 	[vaganov corrected]:
:
\begin{itemize}
    \item свойства изделий, полученных методом СЛС схожи со свойствами изделий полученных традиционными методами, такими как extrusion, injection molding, hot pressing, etc.
    \item теоретически, в технологи СЛС может быть использова любой материал, который можно получить в порошке и расплавить при increasing temperature
    \item при печати сложых продуктов не нужно дополнительных опорных элементов.
\end{itemize}

%Про неисследованность
	Researches on the application of other polymers in SLS technology are very limited. In addition to polyamide, polystyrene and polycarbonate are used as a starting material for SLS.12,13
	[vaganov corrected]
	
	"Анализ современной литературы показывает, что на сегодняшний день исследования в области применения термостойких полимеров для СЛС сосредоточе-ны лишь на семействе полиэфиркетонов [Berretta S., Ghita O., Evans K. E. // European Polymer Journal, 2014. Vol. 59. P. 218–229]. На данный момент практически отсутствуют сведения об использовании других клас-сов термостойких полимеров, таких, например, как полиимиды (ПИ). Кроме того, существует ряд проти-воречивых сведений о влиянии тех или иных параметров на процесс СЛС и конечные свойства материала в целом. В связи с этим, с научной и технической точки зрения весьма перспективным представляется иссле-дование закономерностей формирования структуры при СЛС порошков ПИ, изучение процесса их сплав-ления в зависимости от свойств порошка, содержания нанонаполнителей, а также параметров СЛС.
\cite{yudin-red}
	
% Про авторов и работы по теме

% Объект и предмет исследования

В работе представлены результаты исследования кристаллической структуры полимерных порошков и квазидвумерных пленок, изготовленных методом селективного лазерного спекания, методом рентгеноструктурного анализа на синхротронном излучении.

%Цели и задачи

%Структура работы
В первой главе ...
Вторая глава ...
Третья глава ...
В заключении ...
