
%АКТУАЛЬНОСТЬ
% Про additive manufacturing

Аддитивное производство, также исвестное как 3D-печать, позволяет переводить виртуальные трехмерные модели в физические объекты. С помощью срезов CAD, данных трехмерного сканирования или томографии, 3D-принтер строит объекты слой за слоем без помощи прессформ или обработки на станке. 3D-печать делает возможным децентрализованное производство объектов по индивидуальным требоаваниям, используя цифровое хранилище данных или интернет. 

Полимерны являются наиболее часто используемыми материалами для 3D-печати. Ассортимент полимеров, используемых в аддитивном производстве, охватывает термопласты, реактопласты, эластомеры, гидрогели, функциональные полимеры, композиты и биологические системы \cite{3d-review}. 

С развитием аддитивных технологий от инструментов прототипирования к производству реальных деталей для конечного использования, возникает растущая потребность в возможности  обработки много большего числа различных материалов, чем возможно в настоящий момент. Материалы, наиболее часто используемые в 3D-печати, такие как полиамид-12, не удовлетворяет требованиям, предъявляемым к большинству коммерческих продуктов. Таким образом, как академические, так и индустриальные учреждения проявляют значительный интерес к увеличению ассортимента доступных материалов и исчерпывающему пониманию их фундаментальных свойств \cite{sls-material}.

Примером быстрорастущего сектора, активно использующего аддитивные технологии, является аэрокосмическая индустрия, производство для которой приносит 18.2\% всех доходов индустрии аддитивного производства \cite{avia}. При этом одной из основных технологий производста является селективное лазерное спекание (СЛС). Многообещающим направлением в СЛС является использование  термостабильных термопластичых полимеров c высокими характеристиками и модификация полимеров добавками наночастиц. Термопласты хорошо подходят для этой технологии благодаря относительно низким температурам плавления \cite{conditions}. Например, из композитов нейлона изготавливают детали двигателей, требующие большой теплостойкости \cite{avia}.
  	"Анализ современной литературы показывает, что на сегодняшний день исследования в области применения термостойких полимеров для СЛС сосредоточе-ны лишь на семействе полиэфиркетонов [Berretta S., Ghita O., Evans K. E. // European Polymer Journal, 2014. Vol. 59. P. 218–229]. На данный момент практически отсутствуют сведения об использовании других клас-сов термостойких полимеров, таких, например, как полиимиды (ПИ). 
  Кроме того, существует ряд проти-воречивых сведений о влиянии тех или иных параметров на процесс СЛС и конечные свойства материала в целом. В связи с этим, с научной и технической точки зрения весьма перспективным представляется иссле-дование закономерностей формирования структуры при СЛС порошков ПИ, изучение процесса их сплав-ления в зависимости от свойств порошка, содержания нанонаполнителей, а также параметров СЛС.
\cite{yudin-red}
  
 В ИВС РАН были синтезированы термопластичные частично кристаллические полиимиды (ПИ) гомологического ряда Р-ОДФО на основе отечественного резорцинового диангидрида Р и четырехядерного диамина ОДФО  \cite{yudin-red}. Изучается влияние условий синтеза и параметров СЛС на характеристики изделий из данного полиимида. Для более полного понимания свойств материала необходимо изучить также и кристаллическую структуру как начального порошка, так и образцов, изготовленных из него по технологии СЛС

В работе представлены результаты исследования кристаллической структуры  порошков полимера Р-ОДФО, а также пленок, изготовленных тз него методом селективного лазерного спекания.

В первой главе изложены краткие сведения о технологии селективного лазерного спекания, рассмотрены характеристики порошков, важные для использования этой технологии, и сведения о кристаллических структурах в полимерных материалах.
Во второй главе рассматриваются раздичные методы исследования кристалличности, а также описывается методика проведения эксперимента по дифракции рентгеновского излучения, позволяющего 
Третья глава посвящена обсуждению результатов измерений, алгоритма обработки данных и структуры изученных образцов.
В заключении приводятся выводы и обсуждаются перспективы дальнейшего изучения материалов на основе Р-ОДФО.
