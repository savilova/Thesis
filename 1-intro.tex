
%АКТУАЛЬНОСТЬ
% Про additive manufacturing
Технология 3Д-печати, хоть и набирает популярность, пока что используется в основном для быстрого прототипирования. Разработка материалов и технологий, позволяющих печатать изделия с высокими механическими свойствами, представляет актуальную задачу для индустрии. [3d review]
Additive manufacturing (AM) alias 3D printing translates computer-aided
design (CAD) virtual 3D models into physical objects. By digital slicing of CAD, 3D scan,
or tomography data, AM builds objects layer by layer without the need for molds or
machining. AM enables decentralized fabrication of customized objects on demand by
exploiting digital information storage and retrieval via the Internet. The ongoing transition
from rapid prototyping to rapid manufacturing prompts new challenges for mechanical
engineers and materials scientists alike. Because polymers are by far the most utilized class
of materials for AM,\\
The
range of polymers used in AM encompasses thermoplastics, thermosets, elastomers,
hydrogels, functional polymers, polymer blends, composites, and biological systems.
Aspects of polymer design, additives, and processing parameters as they relate to enhancing build speed and improving accuracy,
functionality, surface finish, stability, mechanical properties, and porosity are addressed. Selected applications demonstrate how
polymer-based AM is being exploited in lightweight engineering, architecture, food processing, optics, energy technology,
dentistry, drug delivery, and personalized medicine. Unparalleled by metals and ceramics, polymer-based AM plays a key role in
the emerging AM of advanced multifunctional and multimaterial systems including living biological systems as well as life-like
synthetic systems.\\
\cite{3d-review}



Нужны метариалы \cite{sls-material}



% Про отечественные аналоги
Такие материалы делает, например, Ultem [ССЫЛКА] Ввиду сами-знаете-чего, актуальна разработка и изучение отечественных аналогов. В ИВС РАН синтезированы термопластичные частично кристаллические ПИ гомологического ряда Р-ОДФО на основе отечественного резорцинового диангидрида Р (1,3-бис-(3,3,4,4-дикарбоксифенокси)-бензол) и четырехядерного диамина ОДФО (4,4-бис(4-аминофенокси)бифенил) методом химической (с использованием катализаторов) и термической имидизации [Юдин В. Е., Светличный В. М. // Высоко-молекулярные соединения, серия С, 2016. Т. 58. №1. С. 19]. Исследованы форма, размер частиц (фракци-онный состав ПИ порошка) \cite{yudin-red}
% Про СЛС
СЛС - кратко [ССЫЛКА]

Применяется для индустрии (3д review)

Таким образом, многообещающим направлением в СЛС является использование high-performance термостабильных термопластичых полимеров и модификация полимеров with small additives наночастиц.
Главные преимущества СЛС перед другими технологиями, это 	[vaganov corrected]:
:
\begin{itemize}
    \item свойства изделий, полученных методом СЛС схожи со свойствами изделий полученных традиционными методами, такими как extrusion, injection molding, hot pressing, etc.
    \item теоретически, в технологи СЛС может быть использова любой материал, который можно получить в порошке и расплавить при increasing temperature
    \item при печати сложых продуктов не нужно дополнительных опорных элементов.
\end{itemize}

%Про неисследованность
	Researches on the application of other polymers in SLS technology are very limited. In addition to polyamide, polystyrene and polycarbonate are used as a starting material for SLS.12,13
	[vaganov corrected]
	
	"Анализ современной литературы показывает, что на сегодняшний день исследования в области применения термостойких полимеров для СЛС сосредоточе-ны лишь на семействе полиэфиркетонов [Berretta S., Ghita O., Evans K. E. // European Polymer Journal, 2014. Vol. 59. P. 218–229]. На данный момент практически отсутствуют сведения об использовании других клас-сов термостойких полимеров, таких, например, как полиимиды (ПИ). Кроме того, существует ряд проти-воречивых сведений о влиянии тех или иных параметров на процесс СЛС и конечные свойства материала в целом. В связи с этим, с научной и технической точки зрения весьма перспективным представляется иссле-дование закономерностей формирования структуры при СЛС порошков ПИ, изучение процесса их сплав-ления в зависимости от свойств порошка, содержания нанонаполнителей, а также параметров СЛС.
\cite{yudin-red}
	
% Про авторов и работы по теме

% Объект и предмет исследования
\textbf{Объект} исследования... Соответственно, \textbf{предмет} исследоания -- морфология и кристаллическая структура.

%Структура работы

