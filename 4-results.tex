	\section{Получение и первичная обработка данных}
	
	Детектирование дифракционных пиков при сканировании образца наноразмерным рентгеновским пучком с шагом 1 мкм позволяет определить наличие как чисто аморфных, так и частичнокристаллических областей.
		\begin{wrapfigure}[19]{r}{0.6\textwidth}
\vspace{-15px}
  \begin{center}
    \includegraphics[width=0.95\linewidth]{fig/obj.png}
    \vspace{3px}
    \caption{Изображение с 2D-детектора}
    \label{fig:difractogram}
  \end{center}
\end{wrapfigure}
	
	 
	Типичное изображение, полученное на детекторе, представлено на рис. \ref{fig:difractogram}. Различие в сигналах кристаллических и аморфных областей лучше видно на фрагментах дифрактограмм в полярной системе координат, представленных на рис. \ref{fig:azim}.
	
		\begin{figure}[t]\center
\begin{tabular}{cc}
\includegraphics[width=0.5\linewidth]{fig/azim-amo.pdf}
&
\includegraphics[width=0.5\linewidth]{fig/azim-cryst.pdf}
\end{tabular}
\caption{Фрагменты дифрактограмм (в координатах $(q,\chi)$) в аморфной (слева) и кристаллической (справа) областях.}
\label{fig:azim}
\end{figure}
	
	
\paragraph{Интегрирование.}

В результате азимутального интегрирования (по $\chi$) получены одномерные профили  (рис. \ref{fig:waxs_profile}) в диапазоне $q$ от $0.5$ до $3$ \AA$^{-1}$
В широкоугловом диапазоне наблюдаются два главных пика.

\begin{figure}[h!]
    \centering
    \includegraphics[width = \linewidth]{fig/waxs_profile.pdf}
    \caption{Одномерный профиль после азимутального интегрирования}
    \label{fig:waxs_profile}
\end{figure}


	\section{Вычитание фона}
	
	В отличие от обычного фонового шума, широкий сигнал рассеяния от аморфной фазы нельзя определить одним для всех измерений, так как его форма зависит от степени кристалличности. Таким образом, его оценку необходимо проводить для каждого профиля отдельно.
	Стандартные алгоритмы для автоматического распознавания фона, основанные на полиномиальной аппроксимации, показали себя не продуктивными в нашем случае. Распознавание кристаллических пиков и аморфного фона производилось с помощью нелинейного фильтра "rolling ball". В рентгеноструктурном анализе он как правило применяется к двумерным дифрактограммам кристаллических материалов, как например, в работе \cite{ball2018}. Однако подходящая реализация для одномерных профилей, необходимая в случае дифракции на полимерах, не представлена в открытых источниках. Ниже (листинг \ref{lst:ball}) приведена реализация алгоритма для одномерных профилей на языке Python. 
	Идеи для простой реализцации позаимствованы из работы \cite{ball-code}. Ввиду малого отношения сигнал-шум в алгоритме также применяется сглаживание фильтром Савитского-Голея.
\vspace{5px}
	\begin{lstlisting}[language=Python, caption=Алгоритм распознавания фона, label={lst:ball}]
	

import numpy as np
 
profile = np.load('profile.npy')
r = 40 #ball radius

t1 = np.zeros(profile.shape[0],dtype=np.float32)
for i in range (t1.shape[0]):
    for j in range(-r,r):
        if ((i+j)>0 and (i+j)<t1.shape[0]):
            
            if(t1[i]<profile[i+j]):
                t1[i]=profile[i+j]
                
t2 = np.full(profile.shape[0], 10000,dtype=np.float32)


for i in range (t2.shape[0]):
    for j in range(-r,r):
        if ((i+j)>0 and (i+j)<t2.shape[0]):
            
            if(t2[i]>profile[i+j]):
                t2[i]=profile[i+j]
t3 = np.zeros(profile.shape[0],dtype=np.float32)
count = np.zeros(profile.shape[0],dtype=np.float32)
back = np.zeros(profile.shape[0],dtype=np.float32)

for i in range(t3.shape[0]): #smooth
    for j in range(-r,r):
        if ((i+j)>0 and (i+j)<t3.shape[0]):
            t3[i]+=t2[i+j] #sum
            count[i]+=1
            
    back[i] = t3[i]/count[i] #average
\end{lstlisting}
\vspace{5px}

	Принцип действия алгоритма проиллюстрирован на рис. \ref{fig:ball}. 
	Действие алгоритма можно представить как круг заданного радиуса, который "катится" по профилю.Траектория его центра образует линию, которая и вычитается из начального профиля. Пики, чья ширина меньше радиуса круга, не вычитаются, и остаются в конечном профиле. Так, алгоритм позволяет убирать широкие пика рассеяния аморфной фазы, и оставлять только узкие кристаллические пики.
	
	
	\begin{figure}[ht]
	    \centering
	    \includegraphics[width=\linewidth]{fig/ball.PNG}
	    \caption{Caption}
	    \label{fig:ball}
	\end{figure}



	\section{Аппроксимация пиков}
	
	Как видно из рис. \ref{fig:fitting}, кристаллические пики хорошо аппроксимируются контуром Фойгта, как и полагается пикам рентгеновского рассеяния на кристаллах.
	
		\begin{figure}[ht]\center
\begin{tabular}{rcl}
\includegraphics[width=0.33\linewidth]{fig/fitGaussian.pdf}
&
\includegraphics[width=0.33\linewidth]{fig/fitGaussian.pdf}
&
\includegraphics[width=0.33\linewidth]{fig/fitGaussian.pdf}
\end{tabular}
\caption{Фиттинг}
\label{fig:fitting}
\end{figure}
	

	

	\section{Картография образцов}
	\paragraph{Нормализация.}
	
	\paragraph{Частицы порошков.}Составление карт кристалличности образцов показывает, что сами порошки состоят из частичнокристаллических частиц заявленных размеров.
	
	\begin{figure}[h]
	    \centering
	    \begin{tabular}{ccc}
\includegraphics[width=0.33\linewidth]{fig/powder_optic.png}
&
\includegraphics[width=0.33\linewidth]{fig/powder_dif1.png}
&
\includegraphics[width=0.33\linewidth]{fig/powder_dif2.png}
\end{tabular}
	    \caption{Отдельные частицы}
	    \label{fig:powder}
	\end{figure}
	
	\paragraph{Область SAXS.}
	Если проинтегрировать область малоуглового рассеяния, можно получить представление о морфологии частиц.
	
	\paragraph{Интенсивность в области WAXS.}
	
	
	
		\begin{figure}[ht]\centering
\begin{tabular}{cc}
\includegraphics[width=0.5\linewidth]{fig/map-1.png}
&
\includegraphics[width=0.5\linewidth]{fig/map-2.png} \\
\includegraphics[width=0.5\linewidth]{fig/map-1.png}
&
\includegraphics[width=0.5\linewidth]{fig/map-2.png}
\end{tabular}
\caption{Карты кристалличности}
\end{figure}
	
	
	
	
\section{Расчет характеристик}
