	\section{Получение и первичная обработка данных}
	
	Детектирование дифракционных пиков при сканировании образца наноразмерным рентгеновским пучком с шагом 1 мкм позволяет определить наличие как чисто аморфных, так и частичнокристаллических областей.
		\begin{wrapfigure}[19]{r}{0.6\textwidth}
\vspace{-15px}
  \begin{center}
    \includegraphics[width=0.95\linewidth]{fig/obj.png}
    \vspace{3px}
    \caption{Изображение с 2D-детектора}
    \label{fig:difractogram}
  \end{center}
\end{wrapfigure}
	
	 
	Типичное изображение, полученное на детекторе, представлено на рис. \ref{fig:difractogram}. Различие в сигналах кристаллических и аморфных областей лучше видно на фрагментах дифрактограмм в полярной системе координат, представленных на рис. \ref{fig:azim}.
	
		\begin{figure}[t]\center
\begin{tabular}{cc}
\includegraphics[width=0.5\linewidth]{fig/azim-amo.pdf}
&
\includegraphics[width=0.5\linewidth]{fig/azim-cryst.pdf}
\end{tabular}
\caption{Фрагменты дифрактограмм (в координатах $(q,\chi)$) в аморфной (слева) и кристаллической (справа) областях.}
\label{fig:azim}
\end{figure}
	
	
\paragraph{Интегрирование.}

В результате азимутального интегрирования (по $\chi$) получены одномерные профили  (рис. \ref{fig:waxs_profile}) в диапазоне $q$ от $0.5$ до $3$ \AA$^{-1}$
В широкоугловом диапазоне наблюдаются два главных пика.

\begin{figure}[h!]
    \centering
    \includegraphics[width = \linewidth]{fig/waxs_profile.pdf}
    \caption{Одномерный профиль после азимутального интегрирования}
    \label{fig:waxs_profile}
\end{figure}


	\section{Вычитание фона}
	
	Стандартные алгоритмы для автоматического распознавания фона, основанные на полиномиальной аппроксимации, показали себя не продуктивными в нашем случае. Распознавание кристаллических пиков и аморфного фона производилось с помощью нелинейного фильтра "rolling ball". В рентгеноструктурном анализе он как правило применяется к двумерным дифрактограммам кристаллических материалов, как например, в работе \cite{ball2018}. Однако подходящая реализация для одномерных профилей, необходимая в случае дифракции на полимерах, не представлена в открытых источниках. Ниже (листинг \ref{lst:ball}) приведена реализация алгоритма для одномерных профилей на языке Python. 
	Идеи для простой реализцации позаимствованы из работы \cite{ball-code}. 

	\begin{lstlisting}[language=Python, caption=Алгоритм распознавания фона, label={lst:ball}]
	
import numpy as np
 
def incmatrix(genl1,genl2):
    m = len(genl1)
    n = len(genl2)
    M = None #to become the incidence matrix
    VT = np.zeros((n*m,1), int)  #dummy variable
 
    #compute the bitwise xor matrix
    M1 = bitxormatrix(genl1)
    M2 = np.triu(bitxormatrix(genl2),1) 
 
    for i in range(m-1):
        for j in range(i+1, m):
            [r,c] = np.where(M2 == M1[i,j])
            for k in range(len(r)):
                VT[(i)*n + r[k]] = 1;
                VT[(i)*n + c[k]] = 1;
                VT[(j)*n + r[k]] = 1;
                VT[(j)*n + c[k]] = 1;
 
                if M is None:
                    M = np.copy(VT)
                else:
                    M = np.concatenate((M, VT), 1)
 
                VT = np.zeros((n*m,1), int)
 
    return M
\end{lstlisting}

	Принцип действия алгоритма проиллюстрирован на рис. \ref{fig:ball}. 
	
	\begin{figure}[ht]
	    \centering
	    \includegraphics[width=\linewidth]{fig/ball.PNG}
	    \caption{Caption}
	    \label{fig:ball}
	\end{figure}



	\section{Аппроксимация пиков}

	\section{Карты кристалличности}
	
	\section{Параметры решетки}
	
	

\begin{figure}
    \centering
    \includegraphics[width=\linewidth]{fig/gauss-fit.png}
    \caption{Аппроксимация формы пика по разным моделям}
    \label{fig:my_label}
\end{figure}
	
	\paragraph{Карты кристалличности}
	
	\begin{figure}[ht]\center
\begin{tabular}{cc}
\includegraphics[width=0.5\linewidth]{fig/map-1.png}
&
\includegraphics[width=0.5\linewidth]{fig/map-2.png} \\
\includegraphics[width=0.5\linewidth]{fig/map-1.png}
&
\includegraphics[width=0.5\linewidth]{fig/map-2.png}
\end{tabular}
\caption{Карты кристалличности}
\end{figure}
	
	

	
	
	
	Свойства и результаты прочих исследований:
	[Vaganov corrected]