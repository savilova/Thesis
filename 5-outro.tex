\section{Результаты и выводы}

По итогам работы получены следующие результаты:

\begin{itemize}
    \item С помощью оптической микроскопии установлена морфология и размер частиц порошка. Синтезированные полимернные частицы имеют размеры $4.5 \pm 2.6$ мкм ($N = 731$), оптимальные для 3D-печати методом селективного лазерного сплавления. 
    \item В сканирующем дифракционном эксперименте получены дифрактограммы для 2 образцов порошков и 2 спеченных пленок, позволяющие получить пространственно разрешенную информацию о кристаллической структуре.
    \item Предложена реализация алгоритма автоматиеского распознавания сигнала аморфной фазы. 
    \item По данным измерений построены карты кристалличности порошков и пленок для областей размером порядка 100х100 мкм$^2$
    \item Подтверждена частичная кристалличность исходных частиц порошка.
    \item По картам пленок установлено, что при СЛС имеет место частичное плавление и консолидация частиц по механизму жидкофазного спекания
\end{itemize}


выводы

Можно много узнать из таких карт.
Как видно из данных механики (ссылка на отчет), пористость надо уменьшать, находить способы лучше из связывать.

\section{Планы на будущее}

Дальнейшие исследования будут касаться композитных материалах на основе изучаемого полиимида Р-ОДФО, и их применения для лазерного спекания.
Присутствие нанодобавок во многих случаях приводит к уменьшению пористости и улучшению механических свойств \cite{sls-compositeб comp-review}. В частности, композитный материал на основе Р-ОДФО, усиленный углеродными нановолокнами, показывал улучшенню кристалличность и прочность по сравнению с чистым полиимидом \cite{pi-formula}.

Важно будет изучить влияние добавок на кристаллизацию и морфологию кристаллических структур. Также, есть основания предпологать, что с помощью добавок возможно улучшить поглощение излучение композиционным материалом, тем самым оптимизировав процесс производства.



\section{Благодарности}
Выражаю искреннюю благодарность своему научному руководителю Иванову Дмитрию Анатольевичу, за ; Родыгину Александру Игоревичу и Графской Ксении Николаевне;
Юдину Владимиру Евгеньевичу, ИВС РАН, за приглашение в совместный проект
Российскому фонду фундаментальных исследований (РФФИ) за финансовую поддержку проекта.

Отдельная благодарность ОСПС МФТИ. 